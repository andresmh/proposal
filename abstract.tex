%abstract.tex
%Abstract for AYB PhD thesis

% The abstractpage environment sets up everything on the page except
% the text itself.  The title and other header material are put at the
% top of the page, and the supervisors are listed at the bottom.  A
% new page is begun both before and after.  Of course, an abstract may
% be more than one page itself.  If you need more control over the
% format of the page, you can use the abstract environment, which puts
% the word "Abstract" at the beginning and single spaces its text.

%% You can either \input (*not* \include) your abstract file, or you can put
%% the text of the abstract directly between the \begin{abstractpage} and
%% \end{abstractpage} commands.

% First copy: start a new page, and save the page number.
%\cleardoublepage
% Uncomment the next line if you do NOT want a page number on your
% abstract page.
% \pagestyle{empty}
\setcounter{savepage}{\thepage}
\begin{abstractpage}
\addcontentsline{toc}{chapter}{Abstract}
\setlength{\parskip}{0pt} %Assumes 11pt type
\begin{spacing}{1}
In this dissertation proposal I describe a framework to design and study an online community of amateur creators.
I focus on remixing as the lens to understand the social, cultural and technical infrastructure of a social media environment that supports creative expression.
I am motivated by three broad questions:
1) what are the \emph{structural} properties of an online remixing community?
2) what is the \emph{functional} role of remixing in cultural production and social learning?
3) what are amateur creators' \emph{attitudes} towards remixing?
As part of this work, I conceived, developed and studied the Scratch Online Community: a website where young people share and remix their own video games and animations, as well as those of their peers.
In three years, the community has grown to close to 800,000 registered members and more than 1.7 million community-contributed projects.
\end{spacing}
\setlength{\parskip}{11pt plus3pt minus3pt} %Assumes 11pt type
\end{abstractpage}

%\cleardoublepage
