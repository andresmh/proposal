I investigate remixers' and originators' attitudes toward remixing, in particular, I analyze how participants use or perceive remixing as a cooperative or even 'antisocial' practice and how the system design may influence these attitudes. From the originator's perspective, I look at how and under what circumstances he or she reacts with: 
1) indifference, 
2) acceptance, 
3) encouragement, 
4) conditional acceptance or on an 
5) oppositional attitude. 
Similarly, I look at how remixers may go about remixing: 
1) obliviously, without regard to the norms, 
2) cautiously, or even 
3) antagonistically, as a form of trolling.
I explore these perspectives through case studies and by analyzing people's responses to design interventions intended to support proremixing attitudes and how failures to do so help inform future design repetitions. Analyzing these responses and attitudes can serve as a metric to understand the health of the community and to motivate further design interventions.
