\subsection{Attitudes Toward Remixing}

I investigate remixers' and originators' attitudes toward remixing. In particular, I analyze how participants perceive remixing and how they do or do not cooperate by letting or encouraging others to reuse their work.
This analysis is driven by an interest in understanding how the system design may influence these attitudes. 
I plan to analyze these issues from the perspective of people whose projects are remixed as well as from those creating the remixes.
 
Previous work studying people's attitudes toward remixing in the Scratch Online Community \citep{hill_responses_2010, monroy-hernandez_computers_2011} found evidence that people are as likely to react negatively as they are to react positively when someone remixes their work. 
Additionally, we have found originators are more likely to respond negatively when their projects are more complex.

Broadly speaking, people whose projects are remixed react either by being indifferent to it, accepting it, conditionally accepting it (for example, specific rules or norms have to be followed), or by explicitly opposing to it (for example, posting a negative reaction comment like ``you stole my project!'').
Similarly, remixers go about remixing by either being oblivious of the norms (for example, giving credit or asking for permission has emerged as a norm in the  community), or cautiously doing it by asking for permission first, or even being confrontational and using remixing as a form of ``trolling'' \citep{donath_identity_1998}.

\subsubsection{Proposed Work}

I plan to analyze people's attitudes toward remixing in the Scratch Online Community through case studies and experiments.
Additionally, I expect these metrics of people's responses and attitudes will serve to understand the health of the community and to motivate further design interventions.
In particular, I propose two studies for studying Scratch participant's attitudes toward remixing that complement the two recently published articles \citep{monroy-hernandez_computers_2011,hill_responses_2010}.


% NOTES The overarching questions are: How do young people respond to remixing? How are these attitudes represented in the community? When do they embrace remixing and when do they reject it?  I have and will analyze young people’s attitudes based on their words (interviews) comments, reports (flags) and strategies for deterring (obfuscation, pseudo-licenses, Vigilantism) or supporting remixing (commenting, scaffolding, framework approach, creation of narratives).
% Crowding out remixing by featuring
% Evolution of reactions as design changes

\emph{Plasticity of Virtue.}
For the past five months, I have been running an experiment aimed at ascertaining how people's attitudes toward remixing could be changed through a design intervention.
The study consists of sending notification messages that try to appeal to various motivations for cooperating in the Scratch online community. 
As mentioned before, a contentious issue in the Scratch community has been the acceptance of remixing by those whose projects are remixed. 
People only learn about remixes of their projects by browsing their project pages and looking at the list of derivative works. 
The notification page is the principal form of communication between the system and the users.
The experiment consists of informing people when one of their projects gets remixed.
This experiment provides an opportunity to test: 1) the comparative effectiveness of various ways of communicating the same event and 2) the permanency of the behavioral change, if any. 

The experiment consists of two phases.
The first is the notification period. 
When someone's project gets remixed, the system automatically assigns the creator of that project to one of the treatment conditions. 
From then on, the person receives messages whenever someone remixes his or her project. The message depends on the treatment category.
For example some messages are neutral (for example, ``Your project FooBar was remixed. Check out the remix.''); 
others are positive (for example,``Congratulations! Your project FooBar was remixed. Check out the remix .'');
others try to elicit generosity (for example, ``Your project FooBar has been remixed. Sharing your work is a generous thing to do and a great thing for the Scratch community! Check out the remix.'').
Other messages try to elicit conformity, reputation building and fairness.
Finally, a control category is added where no notification is sent.

The second phase of the experiment is sending no notifications. In this phase of the experiment, the notifications stop and people's reactions continue to be logged.

After a few weeks we look at the results of pre- and post-phase one, to analyze the effectiveness of each treatment by measuring the number of positive and negative reactions that originators leave on the remixes. 
Additionally, I analyze the pre- and post-phase two, to see how malleable the behavior is.

\emph{Featuring Top Remixes.}
The home page of the Scratch website now has a section called ``What the Community is Remixing''  that features the three top remixed projects in the past two weeks.
This section did not exist when the website was first unveiled.
In fact, this section was added as measure to counter the backlash against remixing by showing that getting one's project remixed could increase social status in the community (being on the front page is an important reputation-building mechanism).
This study aims at assessing whether a design intervention aimed at increasing the acceptance of remixing had a ``crowding out'' effect by decreasing the quality or effort people put into making their remixes.
I have devised metrics to operationalize the complexity and effort a creator puts into creating his or her project. These metrics are based on the number of sprites, scripts, blocks, diversity of blocks and time it took from the first save to the hard disk until it gets shared on the website.

