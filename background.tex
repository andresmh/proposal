\chapter{Background}

Remixes, mash-ups, forks and derivative works are terms used to refer to a form of social production based on the idea of creating something out of existing creations.
Wikipedians editing one another's articles, Free and Open-Source programmers reusing existing code libraries, and teenagers mixing their favorite videos on YouTube, are all examples of extensive remixing practices found today in the digital landscape.

% TODO: add more references about the intersection of community and remixing
I build on existing work that places remixing in new forms of economic and cultural production \citep{benkler_wealth_2006, jenkins_convergence_2006,manovich_remix_2005,sinnreich_ethics_2009}, and describes some challenges that remixing poses to our existing legal and moral assumptions \citep{lessig_remix:_2008, posner_little_2007}.

% SELF: My work will build more on new media literacy than on situated learning so I am mentioning this first.
Furthermore, I build on work that advocates remixing as a new media literacy skill \citep{ito_hanging_2010, jenkins_confronting_2009, livingstone_taking_2008, perkel_copy_2008} and that positions remixing as legitimate form of participation in social learning environments, building on learning philosophies such as Constructionism \citep{papert_mindstorms_1980} and Situated Learning \citep{lave_situated_1991}.

Finally, I look at the implications of remixing for the design of social computing systems. I do this by connecting to research on human-computer interaction that examines the participation patterns and the design of peer-production communities that engage people in sharing and remixing videos \citep{diakopoulos_evolution_2007,shaw_community_2006}, images \citep{seneviratne_policy-aware_2009}, music \citep{cheliotis_analysis_2009} and hypertext \citep{viegas_studying_2004}.

\section{Economic and Cultural Production}

Network information technologies have enabled social production to emerge as a new form of economic production parallel to markets and firms.
People engaging in this new form of production, called Commons-based Peer Production \citep{benkler_coases_2002}, rely on the availability of existing common resources that are often re-purposed for the creation of new products that go back to that communal source for others to reuse as well.

Culture has also been influenced by these new forms of peer-to-peer participation where the boundaries between producers and consumers blur, what \citet{jenkins_convergence_2006} calls Participatory Culture.

These creative practices, based on the idea of building new things by combining existing ones, are not necessarily new. 
Artists, especially postmodernist, have engaged in similar practices through ``appropriation art'', ``pastiche'', ``collage'', ``sampling'' and ``bricolage''. 
Furthermore, folk culture and oral traditions rely on this idea of re-creating and remixing what others have made. %Citation need
For example, as \citet{manovich_remix_2005} argues, remixing is a part of cultural evolution as one can see how ancient Rome was a remix of ancient Greece.
Despite this common feeling that everything old is new again, the influence of digital technologies in remixing-like practices is undeniable. 
These technologies have enabled people to a) create perfect copies and b) go beyond just being inspired by other's creations but to remix the original works themselves \citep{sinnreich_ethics_2009}.

\section{Learning and New Media Literacy}

Remixing is a social activity by nature. Hence, some discussions around remixing and learning stem from learning theories that look at the social context in which learning happens.
Situated Learning is one of these models that has advocated legitimizing peripheral forms of participation, in particular in apprenticeship-like learning environments \citep{lave_situated_1991}. 
This apprenticeship model is also central to Constructionism \citep{papert_mindstorms_1980}, a learning philosophy based on the idea that some of the most valuable learning experiences occur when children engage in building personally meaningful objects in a social context. 
% TODO: Look at how Bruckman cites this in the social context
Building on this theory, \citet{turkle_epistemological_1990} argued for ``bricolage'' as a legitimate learning style, alternative to planning, where learners ``construct theories by arranging and rearranging, by negotiating and re-negotiating with a set of well-known materials'', a model that fits well with the process of remixing.

Similarly, \citet{wenger_communities_1998} stressed the importance of also having access to a ``shared repertoire of communal resources'' to help shape a ``community of practice.''

More recently, cultural anthropologists and media scholars have documented the ways young people engage in creative practices through remixing. 
For example, \citet{ito_technologies_2007} described how children relate to media franchises, such as Pókemon.
She found that children not only consume Pókemon but they engage in remixing and re-creating it, ``demonstrating that children can master highly esoteric content, customization, connoisseurship, remixing.''
\citet{livingstone_taking_2008} has described similar practices concerning Internet use and presented the challenges and possibilities that these creative practices promise.
Similarly, \citet{jenkins_convergence_2006} has narrated how children have become active participants in media creation by remixing their favorite literary characters in Harry Potter and creating rich fan fiction.
Later, \citet{jenkins_confronting_2009} argued that the ability to remix, is a core competency that children must possess to be fluent with new media.

\section{Ethical and Legal Challenges}

Creative appropriation or remixing has been confronted on moral and legal grounds.
In his description of children's fan fiction, \citet{jenkins_convergence_2006} describes how the young girl who created a Harry Potter fan fiction website, was challenged by the company who owns the legal rights to the book and the movie, on the grounds of copyright infringement.
Eventually the company dropped the legal action and reached an agreement.

More recently, appropriation art has been under legal scrutiny \citep{greenberg_art_1992,landes_copyright_2000}. 
For example, a federal court judge determined that photographer Richard Price violated Patrick Cariou's copyrights for remixing a set of pictures by putting them on frames, ``painting over some portions'' \citep{batts_patrick_2011}.

Similarly, countless video remixes on the video-sharing website YouTube have been removed under the Digital Millenium Copyright Act, for being identified as remixes of commercial videos \citep{seneviratne_remix_2010}.

\citet{posner_little_2007} has articulated how plagiarism is highly context-dependent but that one can assess the ethics of appropriation by thinking through issues of deception, perception and social expectation.
\citet{benkler_wealth_2006} argues that if we want peer-production to flourish, we must figure out how to enable remixing:
``If we are to make this culture our own, render it legible, and make it into a new platform for our needs and conversations today, we must find a way to cut, paste, and remix present culture. And it is precisely this freedom that most directly challenges the laws written for the twentieth-century technology, economy, and cultural practice.''

Similarly, \citet{lessig_remix:_2008} has reported on cases of ``copyright extremism'' that, he argues, ``chill''
 innovation and creativity, especially among young people who often engage in these practices.
In response to such copyright extremism, the Creative Commons licenses allow creators to have more control of their copyright and release their work under more permissive licenses that would foster amateur creativity.

At the core of examining the ethics of remixing lies the understanding of cooperation, that is, the idea of how much people are willing to sacrifice their selfish and rational desires, to obtain monetary or reputation gains, to behave in altruistic and cooperative ways. 
These questions lay at the core of literature on human cooperation which is beginning to be translated to social system design.

\section{Social Computing System Design}

Human-computer interaction research has studied the use and design of social computer systems that foster cooperative practices that allow remixing to take place.
% Wikipedia
Wikipedia has been perhaps the most widely researched of those systems. 
For example, \citet{viegas_studying_2004} developed a visualization of Wikipedia edits that led to insights into the nature of the system and its editors' ability to collaborate. 
Later \citet{kittur_harnessing_2008} studied the quality of Wikipedia's articles in relationship to various types of coordination mechanisms.
% Open Source
Similarly, analyses on open source software development have led to insights into the mechanisms that lead to successful cooperative projects. 
\citet{raymond_cathedral_1999}, for example, argued that one of the lessons to be learned from open source software programmers, is the importance of knowing what to rewrite and reuse. 
He describes how Linus Trovalds (the creator of Linux) did not ``try to write Linux from scratch'' instead ``he started by reusing code and ideas from Minix, a tiny Unix-like OS for PC clones.'' 

% Tools
Researchers have also developed web mash-up tools that allow people to remix web content \citep{bolin_automation_2005,wong_making_2007}.
A study on one of those web mash-up tools found that many of its users lacked programming skills and found web mash-up an effective way of searching and aggregating information \citet{nan_zang_whats_2008}.

% Media sharing and remixing commmunities
More specifically on online communities for remixing, \citet{shaw_community_2006} developed a video remixing platform and studied the nature of the most generative video segments intending to understand how to integrate automatic recommendation systems with user-driven suggestions.
\citet{diakopoulos_evolution_2007} analyzed user's participation in a video remixing website and documented how participants developed specific norms for appropriating other people's work that were not encoded in the architecture of the web site.
Additionally, a study of the music remixing online community ccMixter.org, looked at the impact of a remixing contest in the community dynamics.
The study found that the contests increased participation among new comers but that they did not continue using the website after the contest \citep{cheliotis_analysis_2009}. 

