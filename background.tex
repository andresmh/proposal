\chapter{Background}

% Summary of this chapter
% TODO: should this be at the end of the section?

I build on existing work that places remixing in new forms of cultural and economic production \citep{benkler_wealth_2006,jenkins_convergence_2006,manovich_remix_2005,sinnreich_ethics_2009}, and describes some challenges that remixing poses to our existing legal and moral assumptions \citep{lessig_remix:_2008, posner_little_2007}.

Inspired by Situated Learning (e.g. \citet{lave_situated_1991}), I investigate remixing as a legitimate form of participation in a social learning environment.
Furthermore,  I build on the work that advocates remixing as a new media literacy skill \citep{ito_hanging_2010, jenkins_confronting_2009, livingstone_taking_2008, perkel_copy_2008}.
I look at the implications of remixing for the design of social computing systems by connecting to existing human-computer interaction research on systems for remixing and sharing videos \citep{diakopoulos_evolution_2007}, images \citep{seneviratne_policy-aware_2009}, music \citep{cheliotis_analysis_2009} and hypertext \citep{viegas_studying_2004}.

% Is following paragraph better as an intro?
Remixes, mash-ups, forks and derivative works are all names found in peer-production websites that refer to a form of social creativity based on the idea of building something new out of something old.
From Wikipedians editing each other's articles, to open source programmers reusing existing code, to teenagers mixing their favorite videos on YouTube, these are all examples of the wide range of practices where remixing plays a central role.

\section{Cultural and Economic Production}
Network information technologies have enabled social production to emerge as new form of economic production, emerging as an alternative to market and firms-based economies.
People participating in these new economies, create and recreate new things with others using previous contributions to a ``commons''. 
This alternative model is what \citet{benkler_coases_2002} called Commons-based Peer Production, where the ability to remix plays a central role.

% TODO: Insert quote from Wealth of Networks.
Similarly, culture has been influenced by these new forms of peer-to-peer participation where the boundaries between producers and consumers are blurrier, given rise to the the idea that we might live in what \citet{jenkins_convergence_2009} calls the Participatory Culture.
This seemingly new creative practice where building new things by combining existing ones is not necessarily new. 
Artists, especially postmodernist, have engaged in similar practices through ``appropriation art'', ``pastiche'', ``collage'' and ``bricolage''. 
Also, folk culture and oral traditions very much rely on this idea of recreating and remixing what others have made. %Citation needed
\citet{manovich_remix_2005} argues that remixing is an integral part of culture and that one can think of ancient Rome as a remix of ancient Greece.
However, digital technologies have allowed for a  significant change in these age-old remixing practices: they have enabled people to create perfect copies and not only be inspired by other's creations but to remix the original works themselves \citep{sinnreich_ethics_2009}.

\section{Learning through Remixing}
Remixing is a social activity by nature, hence some of the discussions around remixing and learning stem from learning theories that look at the social context in which learning happens.
Situated Learning is one of these models that has advocated for legitimizing peripheral forms of participation, in particular in the context of apprenticeship-like learning environments \citep{lave_situated_1991}. 
This apprenticeship model is also central to Constructionism, a learning philosophy based on the idea that some of the most valuable learning experiences occur when children engage in building personally meaningful objects in a social context \citet{papert}. 

In arguing for this theory and using apprenticeships as a learning model, Papert presents the Brazilian Samba schools as exemplar learning environments. These schools, he argues, allowed both novices and experts to learn from each other by building on each others work and expertise. 
\citet{papert_turkle} later argued for ``bricolage'' as a legitimate learning style, alternative to planning, where learners tinker and recreate previous works.
Similarly,  \citet{wenger} stress the importance of also having access to materials that can help scaffold participation.

Cultural anthropologists and media scholars have documented the ways young people engage in prolific forms of creativity through remixing and have argued not only for its legitimization as a form of learning but also for its support and fostering as a core skill.
For example, \citet{ito_yugi} described how Japanese children relate to media franchises such as Pokemon by not only being consumers of it but also engaging in remixing and recreating them by themselves and with others.
\citet{livingstone_taking_2008} has described similar practices in the UK and presented the challenges and possibilities that these creative practices promise.
Similarly, \citet{jenkins_convergence_2006} has narrated how kids have become active participants in media creation by remixing their favorite literary characters in Harry Potter and creating rich fan fiction.
Finally, \citet{jenkins_confronting_2009} has argued that the ability to remix, is a core competency that children must possess in order to be fluent with new media.

\section{Ethical and Legal Challenges}
Creative appropriation or remixing has been highly controversial and confronted on moral and legal grounds. 
\citet{jenkins_convergence_2006} describes how a young girl who created a Harry Potter fan fiction website was challenged by the company who owns the legal rights to book and the movie on the grounds of copyright infringement. 
More recently, some instances of appropriation art have been under legal scrutiny. For example, photographer Preece was sued for his photographs of Marlborough ads that were large-size replicas of the original ads, his photographs were displayed on the Museum of Modern Art in New York and sold for millions of dollars.
Similarly, he was sued by another photographer for taking his pictures and reusing them in what the original creator deemed outside the Fair Use provision.
Countless number of video remixes on video sharing website YouTube have been taken down after they have automatically been identified as in violation of copyright laws for remixing videos or music. 

Despite the often extreme sides on the issue of piracy and copyright, the law and ethics of remixing are still contested on a case by case basis. Legal scholar, \citet{posner_little_2007} has described how plagiarism is highly context-dependent and has also articulated a framework to assess the morality of different types of content reuse and copying based on issues of social expectations and deception. 
\citet{benkler_wealth_2006} argues that if we want peer-production to flourish, we must figure out how to enable remixing. Similarly, 
\citet{lessig_remix:2008} has narrated the cases of extreme copyright protection that, he argues, ``chill''
 innovation and creativity, especially among young people who engage in this practices more than their adult counterparts. Lessig and others proposed a new set of licenses under the brand of Creative Commons that would allow creators have more control of their copyright and release their work under more permissive licenses that would, at least in theory, foster amateur creativity.

At the core of this discussion is the issue of how much people are willing to sacrifice their ``selfish'' and rational desires to gain economic gains, reputation or attention or to behave in more altruistic ways by relinquishing some of those individual desires for the greater good of a collective. 
These questions lay at the core of literature on human cooperation which is beginning to be translated to social system design.

\section{Social Computing System Design}
The design of social computing systems have been the subject of study for some years in the fields of human-computer interaction and computer supported cooperative work. Some of these studies have focused on understanding what people do on online spaces that support peer-production as well as building systems to test out hypotheses. 
The most widely documented systems where remixing or building on peer's work has flourished are Wikipedia and the Free and Open Source Software movement \citep{raymond, benkler, viegas, newbook, kittur}. 
Researchers have looked, among other things, on the motivations for participating on this spaces as well as possible mechanisms that lead to the success of some of these projects.
More specifically looking at the practice of remixing, the literature is somewhat sparse. 
One study looked at the music remixing site ccMixter.org \citep{cheliotis_analysis_2009}, finding that remixing contests were an effective intervention to increase participation but that the people it attracted did not tend to be the core members of the community and hence were likely to leave after the contest was over.
There is also a qualitative study on a video sharing and remixing website \citet{diakopoulos_evolution_2007} that looked at how participants developed certain norms for appropriating other people's work that were not encoded in the architecture of the website.
\citet{shaw_community_2006} developed a video remixing platform and studied the nature of the most generative video segments looking to understand how to integrate automatic recommendation systems with user-driven suggestions.
\citet{zang_mashup} did a study of mash-up creators finding that many of them lacked programming skills and found web mash-up to be a effective way of searching and aggregating information.
