\chapter{Introduction}

In recent years, there has been an explosion of websites where people share their ideas, creations and daily life.
However, these amateur creators not only share their work, they also remix, remake, re-create, re-tweet, fork, sample, mashup, edit or appropriate other people's work.
Additionally, this creating, sharing and remixing often happens within online communities in what has been called Commons-based Peer Production, Participatory Culture or the Remix Culture.

This new form of cultural and economic production happens in many creative endeavors.
For example, people participate in websites for sharing and remixing \emph{videos} (for instance, YouTube and Vimeo), \emph{images} (for instance, Flickr and DeviantArt), \emph{code} (for instance, GitHub and Sourceforge), \emph{music} (for instance, ccMixter and IndabaMusic), \emph{status updates} (for instance, Twitter and Facebook), \emph{hypertext} (for instance, Wikipedia and Wikia), and even the design of \emph{physical object} (for instance, Ponoko and Etsy).

The Scratch Online Community was designed to also enable amateur creators, especially children aged between eight and sixteen, to share and remix animations and videogames.
In this dissertation, I plan to, first, describe the motivation and sociotechnical design of the Scratch website, then explore how people have used the website to remix one another's work and how the design has influenced these activities.
The goal is to inform designers, practitioners, educators and scholars of the intricacies of designing for remixing.
