\chapter{Introduction}

In recent years, there has been an explosion of online communities where people share their ideas and creations.
Participants of these communities not only share their own work, they also remix, re-create, remake, re-tweet, fork, sample, mash-up, edit or appropriate other people's work to produce derivatives or remixes that are shared back to the community.
Previous work has described how these communities and the activities they support, have far-reaching implications for the way cultural and economic production work.
In this dissertation proposal, I outline a research framework to examine the design and usage of an active online remixing community built to engage young people in social production and creative learning.

Online remixing communities span a wide range of creative endeavors, there are communities for sharing and remixing 
\emph{code} (for instance, GitHub and Sourceforge),
\emph{articles} (for instance, Wikipedia and Wikia), 
\emph{videos} (for instance, YouTube and Vimeo), 
\emph{images} (for instance, Flickr and DeviantArt), 
\emph{status updates} (for instance, Twitter and Facebook),
\emph{music} (for instance, ccMixter and IndabaMusic),
and even the design of \emph{physical objects} (for instance, Ponoko and Etsy).
In this dissertation, I focus on the Scratch Online Community, a website developed to enable amateur creators, especially children aged between eight and sixteen, to share and remix animations and video games.
I plan to, first, describe the motivations and socio technical infrastructure of the Scratch website, then examine how people have used the website to collaborate with others through remixing and how the design has influenced these activities.
